\documentclass{amsart}

\usepackage{epigraph}
\usepackage{natbib}
\usepackage{hyperref}

\begin{document}
\title{Red Queen's Sync Protocol for Ethereum}
\author{Andrew Ashikhmin \& Alexey Akhunov}
\date{March 2019}

\begin{abstract}
TODO: abstract.
\end{abstract}

\maketitle

\epigraph{
    "A slow sort of country!" said the Queen.
    "Now, here, you see, it takes all the running you can do, to keep in the same place.
    If you want to get somewhere else, you must run at least twice as fast as that!"
    }{Lewis Carroll, Through the Looking-Glass and What Alice Found There}

\section{Introduction}
TODO: mention the sync failure problem \cite{akhunov_1x_workshop_part1} and the needs of light clients like Mustekala.
Inspirations like BitTorrent, Parity's warp sync.

\section{Notation}
We mostly follow the conventions and notations of the Yellow Paper \cite{yellow_paper},
for instance $\mathbb{Y}$ denotes the set of nibble sequences.
We use the letter $\pi$: $\pi \in \mathbb{Y}$, $||\pi|| \leq 64$ for prefixes of state or storage trie keys $\mathbf{k} \in \mathbb{B}_{32}$.
A key matches a prefix iff all their first nibbles are the same,
$$\texttt{MATCH}(\mathbf{k}, \pi) \equiv \forall_{i < ||\pi||}: \mathbf{k}'[i] = \pi[i]$$
($\mathbf{k}'$ is a sequence of nibbles, while $\mathbf{k}$ is a sequence of bytes.)

\section{Protocol Specification}
TODO: check geth's fast sync spec.

\section{Suggested Full Sync Algorithm}
TODO: top-level trie with branch nodes only that track blocks; phase 1, phase 2.

\section{Performance Analysis}
For this analysis we assume that all tries are well balanced.
We also assume that all top nodes up to a certain trie depth $d$ are branch nodes, not leaf or extension ones.
This is a reasonable assumption if $d$ is not too big---see~\cite{akhunov_1x_workshop_part2}.

TODO: optimal phase 1 depth.

Let us further assume that all leaf changes are random and independent of each other.
TODO: optimal phase 2 depth.

\section{Conclusion}
TODO: conclusion.

\bibliographystyle{plainnat}
\bibliography{biblio}

\end{document}
